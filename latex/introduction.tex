\section{Introduction}
\label{sec_introduction}
Nowadays, fuel energy is of the most significant type of energy in our society.
In the near future, human might have to face the shortage of fuel energy as a result.
Thus, the application of electric energy is currently part of each nation's development plan.
The electric vehicle is the most cut-edge application of electric energy, which may also be the most common personal transportation in the future society,
since comparing to fuel-consuming vehicles, the electric vehicles are much more environmental friendly and will not be of shortage.

When deploying charging stations, the most significant problem to figure out is to find out the locations of charging stations.
The charging station network should not only guarantee that an electric vehicle can easily access a charing station within its driving range,
but also should cove the city so that an electric vehicle can travel around the whole city after being recharged~\cite{lam2013electric}.

In order to analysis the problem regarding to the network of charging station,
we propose a model of charging station network by means of Graph Theory.
Applying our model, we convert the problem of deploying charging station network to the problem of identifying \textbf{Strongly Connected Graph} and finding \textbf{Induced Subgraph Connected},
which are classic problems in Graph Theory.

This rest of paper is organized as follows.
Section~\ref{sec_profile_us} shows the data of the charging station network in US and how we evaluate the feasibility of migrating personal transportation towards all electric cars by the means of Graph Theory.
Section~\ref{sec_system_architecture} introduces the process architecture of our paper.
Without loss of generality, we decrease the number of factors in this problem. We propose a
In Section~\ref{sec_solve_optimal}, we propose an algorithm based on the Greedy Algorithm to find optimal solution.
Section~\ref{sec_extension} discusses other factors that can shape our model.

Our contribution is three-fold:
\begin{contribution}
We present a novel Two-Layer Process Architecture that suits the problem and simplify factors significantly without loss of generality.
\end{contribution}
\begin{contribution}
We analyze the problem of charging stations network from the view of Graph Theory, and
convert this problem to two kinds of classic problems in Graph Theory.
\end{contribution}
\begin{contribution}
We propose an algorithm to find the optimal solution of charging station network based on the idea of Greedy Algorithm.
\end{contribution}

